\chapter{Theory of drift diffusion modelling}
\label{sec:electrical}

\section{Outline}
OghmaNano's electrical model is a 1D/2D drift-diffusion model (like many others) however the special thing about OghmaNano which makes it very good for disordered materials (Think organics, perovskites and a-Si) is that it goes to the trouble of explicitly solving the Shockley-Read-Hall equations as a function of energy and position space.  This enables one to model effects such as mobility/recombination rates changing as a function of carrier population and enables one to correctly model transients as one does not have to assume all the carriers in the trap states have reached equilibrium.  Things such as ToF transients, CELIV transients etc.. can be modelled with ease. Of course can be used for more ordered materials as well, you then just need to turn the traps off.


\section{Electrostatic potential}
The conduction band/valance band (or LUMO/HOMO in organic semiconductor speak) are defined as

\begin{equation}
E_{LUMO}=-\chi-q\phi
\end{equation}

\begin{equation}
E_{HOMO}=-\chi-E_g-q\phi
\end{equation}

To obtain the internal potential distribution within the device Poisson's equation is solved,

\begin{equation}
\label{eq:pos}
\nabla \cdot \epsilon_0 \epsilon_r \nabla = q (n_{f}+n_{t}-p_{f}-p_{t}-N_{ad}+-N_{ion}+a),
\end{equation}

where $n_{f}$, $n_{t}$ are the carrier densities of free and trapped electrons; $p_{f}$ and $p_{t}$ are the carrier densities of the free and trapped holes; and $N_{ad}$ is the doping density. $N_{ion}$ is the background density of perovskite ions and a is the density of mobile ions.

\section{Free charge carrier statistics}
For free carriers the model can either use Maxwell-Boltzmann statistics i.e.

\begin{equation}
n_{l}=N_c exp \left (\frac{F_n-E_{c}}{kT} \right)
\end{equation}

\begin{equation}
p_{l}=N_v exp \left(\frac{E_{v}-F_p}{kT} \right)
\end{equation}


or full Fermi-dirac statistics i.e.

\begin{equation}
n_{free}(E_{f},T)=\int^{\infty}_{E_{min}} \rho(E) f(E,E_{f},T) dE
\end{equation}

\begin{equation}
p_{free}(E_{f},T)=\int^{\infty}_{E_{min}} \rho(E) f(E,E_{f},T) dE
\end{equation}

where

\begin{equation}
f(E)=\frac{1}{1+e^{{E-E_f}/kT}}
\end{equation}

When using FD statistics free carriers are assumed to move in a parabolic band:

\begin{equation}
\rho(E)_{3D}=\frac{\sqrt{E}}{4\pi^2} \left ( \frac{2m^{*}}{\hbar^2}\right )^{3/2}
\end{equation}

The average energy of the carriers is defined as

\begin{equation}
\label{eq:energy}
\bar{W}(E_{f},T)=\frac{\int^{\infty}_{E_{min}} E \rho(E) f(E,E_{f},T) dE}{\int^{\infty}_{E_{min}} \rho(E) f(E,E_{f},T) dE}
\end{equation}

\input{electrical_srh_intro.tex}
\input{electrical_ss_srh.tex}
\input{electrical_srh.tex}

\section{Free-to-free carrier recombination}
A free-carrier-to-free-carrier recombination (bi-molecular) pathway is also included.  Free-to-free recombination is described using equation \ref{equ:freetofree}

\begin{equation}
R_{free}=k_{r}(n_{f}p_{f}-n_{0}p_{0})
\label{equ:freetofree}
\end{equation}

in some situations where one is trying to fit rate equations to the model it can be useful to have equation \ref{equ:freetofree} written in another form,

\begin{equation}
R_{free}=k_{r}(n_{f}p_{f}-n_{0}p_{0})^{\frac{\lambda+1}{2}}
\label{equ:freetofree_lambda}
\end{equation}

this can be turned on using the option called  \emph{Enable $\lambda$ power in free to free recombination.} in the configure window of the Electrical parameter editor.

Free to free recombination is equivalent to Langevin recombination. However, most organic solar cells have a great deal of trap states and an ideality factor greater than 1.0 suggesting that free-to-free recombination is not the dominant mechanism. See section \ref{sec:the_need_for_trap_states} for a general discussion on the need for trap states and why generally Langevin recombination should not be used in organic solar device models. 

\input{electrical_auger.tex}
\input{electrical_transport.tex}
\input{electrical_perovskite.tex}
\input{interfaces.tex}
\section{Calculating the built in potential}  \label{sssec:initial}
The first step to performing a device simulation, is to calculate the built in potential of the device.  To do this we must know the following things:

\begin{itemize}

  \item The majority carrier concentrations on the contacts $n$ and $p$.
  \item The effective densities of states $N_{LUMO}$ and $N_{HOMO}$.
  \item The effective band gap $E_g$

\end{itemize}

\begin{figure}[H]
\centering
\includegraphics[width=120mm]{./images/bands.png}
\caption{Band structure of device in equilibrium.}
\label{fig:bands}
\end{figure}

\vspace{1em}
The left hand side of the device is given a reference potential of 0 V.  See figure \ref{fig:bands}.  We can then write the energy of the LUMO and HOMO on the left hand side of the device as:

\begin{equation}
E_{LUMO}=-\chi
\end{equation}

\begin{equation}
E_{HOMO}=-\chi-E_{g}
\end{equation}

For the left hand side of the device, we can use Maxwell-Boltzmann statistics to calculate the equilibrium Fermi-level ($F_i$).

\begin{equation}
p_{l}=N_v exp \left(\frac{E_{HOMO}-F_p}{kT} \right)
\end{equation}

We can then calculate the minority carrier concentration on the left hand side using $F_i$

\begin{equation}
n_{l}=N_c exp \left (\frac{F_n-E_{LUMO}}{kT} \right)
\end{equation}

The Fermi-level must be flat across the entire device because it is in equilibrium.  However we know there is a built in potential, we can therefore write the potential of the conduction and valance band on the right hand side of the device in terms of $phi$ to take account of the built in potential.

\begin{equation}
E_{LUMO}=-\chi-q\phi
\label{equ:Ev_rhs}
\end{equation}

\begin{equation}
E_{HOMO}=-\chi-E_g-q\phi
\end{equation}

we can now calculate the potential using

\begin{equation}
n_{r}=N_c exp \left (\frac{F_n-E_{LUMO}}{kT} \right)
\end{equation}
equation \ref{equ:Ev_rhs}.

The minority concentration on the right hand side can now also be calculated using.

\begin{equation}
p_{r}=N_v exp \left (\frac{E_v-F_{HOMO}}{kT} \right)
\end{equation}

The result of this calculation is that we now know the built in potential and minority carrier concentrations on both sides of the device.  Note, infinite recombination velocity on the contacts is assumed.  I have not included finite recombination velocities in the model simply because they would add four more fitting parameters and in my experience I have never needed to use them to fit any experimental data I have come across.

Once this calculation has been performed, we can estimate the potential profile between the left and right hand side of the device, using a linear approximation. From this the charge carrier densities across the device can be guessed.  The guess for potential and carrier densities, is then used to prime the main Newton solver.  Where the real value are calculated.  The Newton solver is described in the next section.


\subsection{Average free carrier mobility}
In this model there are two types of electrons (holes), free electrons (holes) and trapped electrons (holes).  Free electrons (holes) have a finite mobility of $\mu_e^0$ ($\mu_h^0$) and trapped electrons (holes) can not move at all and have a mobility of zero.  To calculate the average mobility we take the ratio of free to trapped carriers and multiply it by the free carrier mobility.:

\begin{equation}
\mu_e(n)=\frac{\mu_e^0 n_{free}}{n_{free}+n_{trap}}
\end{equation}

Thus if all carriers were free, the average mobility would be $\mu_e^0$ and if all carriers were trapped the average mobility would be 0.  It should be noted that only $\mu_e^0$ ($\mu_h^0$) are used in the model for computation and $\mu_e(n)$ is an output parameter.

The value of $\mu_e^0$ ($\mu_h^0$) is an input parameter to the model.  This can be edited in the electrical parameter editor.  The value of $\mu_e(n)$, and $\mu_h(p)$ are output parameters from the model.  The value of $\mu_e(n)$, and $\mu_h(p)$ change as a function of position, within the device, as the number of both free and trapped charge carriers change as a function of position.  The values of  $\mu_e(x)$, and $\mu_h(x)$ can be found in $mu\_n\_ft.dat$ and $mu\_p\_ft.dat$ within the $snapshots$ directory.  The spatially averaged value of mobility, as a function of time or voltage can be found in the files $dynamic\_mue.dat$ or $dynamic\_muh.dat$ within the dynamic directory.

Were one to try to measure mobility using a technique such as CELIV or ToF, one would expect to get a value closer to $\mu_e(n)$ or $\mu_h(p)$ rather than closer to $\mu_e^0$ or $\mu_h^0$.  It should be noted however, that measuring mobility in disordered materials is a difficult thing to do, and one will get a different experimental value of mobility depending upon which experimental measurement method one uses, furthermore, mobility will change depending upon the charge density profile within the device, and thus upon the applied voltage and light intensity.  To better understand this, try for example doing a CELIV simulation, and plotting $\mu_e(n)$ as a function of time (Voltage).  You will see that mobility reduces as the negative voltage ramp is applied, this is because carriers are being sucked out of the device.  Then try extracting the mobility from the transient using the CELIV equation for extracting mobility.  Firstly, the CELIV equation will give you one value of mobility, which is a simplification of reality as the value really changes during the application of the voltage ramp.  Secondly, the value you get from the equation will almost certainly not match either $\mu_e^0$ or any value of $\mu_e(n)$.  This simply highlights, the difficult of measuring $a$ value of mobility for a disordered semiconductor and that really when we quote a value of mobility for a disordered material, it really only makes sense to quote a value measured under the conditions a material will be used.  For example, for a solar cell, values of $\mu_e(n)$ and $\mu_h(n)$, would be most useful to know under 1 Sun at the $P_{max}$ point on a JV curve.


\input{electrical_solverconfig.tex}
\newpage
\section{The need for trap states in device organic models}
\label{sec:the_need_for_trap_states}
Related YouTube videos:
\begin{figure}[H]

\begin{tabular}{ c l }

\includegraphics[width=0.05\textwidth]{./images/youtube.png}

&
\href{https://www.youtube.com/watch?v=2EHfulz7UDU}{Please stop simulating disordered semiconductors without trap states.}

\end{tabular}
\end{figure}
This section explains why trap states need to be considered when simulating disordered materials such as polymer/small molecule devices. It also touches on why using full SRH recombination/trapping model is so important to get physically meaningful results from a device model.

\subsection{The physical and energetic structure of disordered materials.}
Traditional inorganic semiconductor such as crystalline Si or GaAs are highly ordered and are almost completely pure it is not uncommon to get a material that is nine nines pure or, 99.9999999\% pure. Organic semiconductors on the other hand are very really quite dirty with purities often around 99.9\% which is six orders of magnitude more dirty than their inorganic counterparts, thus they have around a million times more impurities than their inorganic counterparts.  Added to this inorganic semiconductors are highly ordered with a regular crystalline structure one can think of them as marbles packed on a solitaire board (see Figure \ref{fig:order}), while organic semiconductors are a floppy mess of molecules which one can think of more as spaghetti bolognese with the spaghetti representing the polymers and the bolognese representing small molecules (see Figure \ref{fig:disorder}).

\begin{figure}[H]
\centering
\begin{tabular}{ c c }

\includegraphics[width=0.5\textwidth,height=0.4\textwidth]{./images/electrical/marbles.jpg}

&
\includegraphics[width=0.5\textwidth,height=0.4\textwidth]{./images/electrical/silicon.png}
\\

\end{tabular}
\caption{Left) Marbles in an ordered arrangement on a solitaire board \cite{image_marbles}; Right) Silicon atoms ordered within a material\cite{image_silicon} Both systems are highly ordered.}
\label{fig:order}
\end{figure}

\begin{figure}[H]
\centering
\begin{tabular}{ c c }


\includegraphics[width=0.5\textwidth,height=0.4\textwidth]{./images/electrical/spaghetti.jpg}

&
\includegraphics[width=0.5\textwidth,height=0.4\textwidth]{./images/electrical/polymer.png}
\\
\end{tabular}
\caption{Left) A plate of spaghetti \cite{image_spaghetti}; Right) A polymer packing like spaghetti. Both systems are highly disordered.}
\label{fig:disorder}
\end{figure}

So on one hand we have an organic material that is messy and highly disordered, and on the other hand we have a material such as silicon that is highly pure and very ordered.  This physical differences results in a very different energetic landscape for the two materials. In the ordered material semiconductor electrons/holes can travel freely in the conduction and valance bands. If an electric field is applied they only experience a small resistive force. Such a band structure is shown in Figure \ref{fig:band_structure}a. In the organic material the picture is very different, due to the disorder and impurities the band structure is full trap states. There are so many trap states that the carriers no longer move freely but hop between the trap states after being thermally excited, such a band structure is shown in shown in Figure \ref{fig:band_structure}b. Thus there are two very different charge transport mechanisms in these two materials.


\begin{figure}[H]
\centering
\begin{tabular}{ c c }


\includegraphics[width=0.5\textwidth,height=0.32\textwidth]{./images/electrical/ordered.png}

&
\includegraphics[width=0.5\textwidth,height=0.32\textwidth]{./images/electrical/disordered.png}
\\
\end{tabular}
\caption{a) The band structure of an ordered semiconductor such as GaAs; b) The band structure of an disordered material such as PM6:Y6 or P3HT:PCBM.}
\label{fig:band_structure}
\end{figure}

\subsection{Trap states and charge density}
[This section needs improving/editing but the sketch of what it should say is there:]
Figure \ref{fig:dos_image} sketches out the distribution of states for Figure \ref{fig:band_structure}. On the left of the image is a ordered semiconductor with a parabolic band structure. The Fermi distribution of electrons is coloured in purple. The right hand side image shows a disordered semiconductor with an exponential density of trap states going into the band gap (sometimes a Gaussian DoS is used).  It can be seen that the DoS and the distribution/energetic position of charge carriers are is very different between the two types of semiconductor.


\begin{figure}[H]
\centering
\begin{tabular}{ c c }

\includegraphics[width=1\textwidth,height=0.4\textwidth]{./images/electrical/band_structure.png}
\\
\end{tabular}
\caption{a) The band structure of an ordered semiconductor such as GaAs; b) The band structure of an disordered material such as PM6:Y6 or P3HT:PCBM.}
\label{fig:dos_image}
\end{figure}

The total charge density at any place in the device can be described by an integration of the Fermi-Dirac function, and the DoS $\rho$.

\begin{equation}
n(E_{f},T)=\int^{\infty}_{E_{min}} \rho(E) f(E,E_{f}) dE
\end{equation}

Where $E_{f}$ is the Fermi level. Clearly $\rho$ will be very different for an ordered and a disordered semiconductor. Thus the dependence of $n(E_{f},T)$ on $E_{f}$ and thus applied voltage will very depending on what $\rho$ is chosen for the device. In practical terms this means that a disordered device will have a lot more traps closer to the Fermi level and thus for any given voltage it will contain one or two orders of magnitude more charge than an ordered device, this can be observed in Charge Extraction experiments. So if one ignores trap states when modelling a disordered device then the function $n(Voltage)$ will be wrong.

If $n(Voltage)$ is not correct then the recombination rate will be wrong for any given voltage:

\begin{equation}
R=k_{r}n(Voltage) p(Voltage)
\end{equation}


Furthermore if $n_{free}(E_{f},T)$ is wrong the mobility will also have an incorrect dependence on voltage:

\begin{equation}
\mu_e(n)=\frac{\mu_e^0 n_{free}}{n_{free}+n_{trap}}
\end{equation}

So if your DoS is wrong (i.e. no traps). Then you have no chance of reproducing a JV curve correctly.  Summary: OghmaNano was written specifically to simulate disordered devices where trap states are play a large role in transport and recombination. Examples of such materials are PM6:Y6 and P3HT:PCBM. OghmaNano includes traps correctly, make sure what ever model you are using also includes traps or it will be wrong.

\subsection{Why you should not use Langevin recombination in device models}
Langevin recombination is defined as,

\begin{equation}
R_{free}=q k_{r}\frac{( \mu_e+ \mu_h) }{2\epsilon_0\epsilon_r} n p
\end{equation}

where $R_{free}$ is the recombination rate, $k_{r}$ is the Langevin reduction factor and all other symbols have their usual meaning. In general Langevin recombination is a bad way to describe recombination in OPV devices. There were some older papers from the early 2010s using this mechanisum but the models could not self consistently describe dark and light JV curves. This is because the mechanism assumes Brownian motion of electrons and holes and that charge carriers of opposite polarity will recombine when they get close enough to fall into each others electrostatic field.  This picture assumes the charge carriers are free and completely neglects the influence of trap states. It was often found that the Langevin equation could not reproduce the experimental results and predicted recombination rates far higher than were experimental observed. To account for this a Langevin reduction factor $k_{r}$ was often introduced into the equation, and a lot of effort went into measuring $k_{r}$. This need for a reduction factor pointed at some deeper issues with the equation.

If we look at the equation for Langevin recombination we can immediately see some issues with it. The first thing we notice is that $R_{free}$ can only ever change as the square of the charge density (n p), but we know from experiment $R_{free}$ can is often a higher order than 2 e.g. $(np)^{1.5}$. Furthermore we can see two mobility terms, however we know from the discussion from above that mobility is a function of carrier density. So the fact that it has the wrong dependence on carrier density and needs a reduction factor points at the mechanism on which it is based being incorrect, and using it will always be like trying to get a square peg in a round hole. 

\subsection{How one can make Langevin recombination work in device models}
So they key problems with Langevin recombination are a wrong dependence on carrier density and the need for a reduction factor. It is possible to make Langevin recombination 'work' by making the charge carrier mobilities a function of carrier density as was done in \cite{mackenzie2011modeling}:

\begin{equation}
R_{free}=q k_{r}\frac{(\alpha \mu_e(n)+\beta \mu_h(n)) n_{tot} p_{tot}}{2\epsilon_0\epsilon_r}
\end{equation}

then by defining a mobility edge and assuming any carrier below the mobility edge could not move and any carrier above it could.  One could define the averaged electron/hole mobility as: 

\begin{equation}
\mu_e(n)=\frac{\mu_e^0 n_{free}}{n_{free}+n_{trap}}
\end{equation}

and

\begin{equation}
\mu_h(n)=\frac{\mu_h^0 p_{free}}{p_{free}+p_{trap}}
\end{equation}

and if one assumes the density of free charge carriers is much smaller than the density of trapped charge carriers one can arrive at

\begin{equation}
R(n,p)=q k_{r}\frac{(\alpha \mu_e^0 n_{free} p_{trap}+\beta \mu_h p_{free} n_{trap}) }{2\epsilon_0\epsilon_r}
\end{equation}

Thus by making the mobility carrier density dependent we arrive at an expression for Langeving recombination that's dependent upon the density of free and trapped carriers (i.e. $n_{free} p_{trap}$ and $ p_{free} n_{trap}$). This is in principle the same as SRH recombination (i.e. a process involving free electrons (holes) recombining with trapped holes (electrons)).  This was a nice simple approach and it worked quite well in the steady state.  However, to make this all work we have to assume all electrons (holes) at any given position in space had a single quasi-Fermi level, which meant they were all in equilibrium with each other.  For this to be true, all electrons (holes) would have to be able to exchange energy with all other electrons (holes) at that position in space and have an infinite charge carrier thermalization velocity.  This is an OK assumption in steady state when electrons (holes) had time to exchange energy, however once we start thinking about things happening in time domain, it becomes harder to justify because there are so many trap states in the device it is unlikely that charge carriers will be able to act as one equilibrated gas with one quasi-Fermi level.  On the other hand the SRH mechanism does not make this assumption, so it is a better description of recombination/trapping.




