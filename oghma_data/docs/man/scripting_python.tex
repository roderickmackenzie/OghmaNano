\section{Python scripting}
\label{sec:pythonscripts}
Related YouTube videos:
\begin{figure}[H]
\begin{tabular}{ c l }

\includegraphics[width=0.05\textwidth]{./images/youtube.png}

&
\href{https://www.youtube.com/watch?v=vyeAzxBZjMg}{Python scripting perovskite solar cell simulation}

\end{tabular}
\end{figure}

There are two ways to interact with \fileext files via python, using native python commands or by using the OghmaNano class structures, examples of both are given below.


\subsection{The native python way}
\label{sec:pythonscripts_native_python}
As described in section \ref{sec:fileformat}, \fileext files are simply json files zipped up in an archive. If you extract the sim.json file form the sim\fileext file you can use Python's json reading/writing code to edit the .json config file directly, this is a quick and dirty approach which will work. You can then use the $os.system$ call to run oghma\_core to execute OghmaNano.

For example were one to want to change the mobility of the 1st device layer to 1.0 and then run a simulation you would use the code listed in listing \ref{python-example}.

\begin{listing}[H]
\begin{minted}[frame=single,framesep=3mm,linenos=false,xleftmargin=21pt,tabsize=4]{python}

	import json
	import os
	import sys

	f=open('sim.json')		#open the sim.json file
	lines=f.readlines()
	f.close()
	lines="".join(lines)	#convert the text to a python json object
	data = json.loads(lines)

	#Edit a value (use firefox as a json viewer
	# to help you figure out which value to edit)
	# this time we are editing the mobility of layer 1
	data['epitaxy']['layer1']['shape_dos']['mue_y']=1.0


	#convert the json object back to a string
	jstr = json.dumps(data, sort_keys=False, indent='\t')

	#write it back to disk
	f=open('sim.json',"w")
	f.write(jstr)
	f.close()

	#run the simulation using oghma_core
	os.system("oghma_core.exe")
\end{minted}
\caption{Manipulating a sim.json file with python and running a OghmaNano simulation.} 
\label{python-example}
\end{listing}

If the simulation in sim.json is setup to run a JV curve, then a file called sim\_data.dat will be written to the simulation directory containing paramters such as PCE, fill factor, $J_{sc}$ and $V_{oc}$.  This again is a raw json file, to read this file in using python and write out the value of $V_oc$ to a second file use the code given in listing \ref{python-example2}.

\begin{listing}[H]
\begin{minted}[frame=single,framesep=3mm,linenos=false,xleftmargin=21pt,tabsize=4]{python}
f=open('sim_info.dat')
lines=f.readlines()
f.close()
lines="".join(lines)
data = json.loads(lines)

f=open('out.dat',"a")
f.write(str(data["Voc"])+"\n");
f.close()

\end{minted}
\caption{Reading in a sim\_data.dat file using Python's native json reader.} 
\label{python-example2}
\end{listing}


\subsection{PyOghma}
One can get a long way by manipulating the OghmaNano json files directly with python as described in \ref{sec:pythonscripts_native_python}. However, using this approach it is not possible (very easy) to run multiple simulations at the same time. And as most modern CPUs have 8 or more cores it seems a waste not to be running multiple simulations when generating large data sets. Furthermore, manipulating json files in Python is not very intuitive and not very python like. For this reason \href{https://github.com/CaiWilliams}{Cai Williams} has written an API called \href{https://github.com/CaiWilliams/PyOghma/tree/master}{PyOghma} which can be used to manipulate OghmaNano json files and also run simulations. This is a stand alone project to OhgmaNano, so direct any questions about this to him!

PyOghma is available on \href{https://github.com/CaiWilliams/PyOghma/tree/master}{GitHub} and also via pip:

\begin{listing}[H]
\begin{minted}[frame=single,framesep=3mm,linenos=false,xleftmargin=21pt,tabsize=4]{bash}
python -m pip install PyOghma
\end{minted}
\end{listing}

An example of using PyOghma is given below \ref{python-example3}:

\begin{listing}[H]
\begin{minted}[frame=single,framesep=3mm,linenos=false,xleftmargin=21pt,tabsize=4]{python}
	import PyOghma as po

	Oghma = po.OghmaNano()
	Results = po.Results()


	source_simulation = "\exapmle\pm6y6\"

	Oghma.set_source_simulation(source_simulation)

	experiment_name = 'NewExperiment'

	Oghma.set_experiment_name(experiment_name)

	mobility = 1e-5
	trap_desnsity = 1e-18
	trapping_crosssection = 1e-20
	recombination_crosssection = 1e-20
	urbach_energy = 40e-3
	temperature = 300
	intensity = 0.5


	experiment_name = 'NewExperiment' + str(1)
	Oghma.clone('NewExperiment0')

	Oghma.Optical.Light.set_light_Intensity(intensity)
	Oghma.Optical.Light.update()

	Oghma.Thermal.set_temperature(temperature)
	Oghma.Thermal.update()

	Oghma.Epitaxy.load_existing()
	Oghma.Epitaxy.pm6y6.dos.mobility('both', mobility)
	Oghma.Epitaxy.pm6y6.dos.trap_density('both', trap_desnsity)
	Oghma.Epitaxy.pm6y6.dos.trapping_rate('both', 'free to trap',..
	trapping_crosssection)
	Oghma.Epitaxy.pm6y6.dos.trapping_rate('both', 'trap to free',..
	recombination_crosssection)
	Oghma.Epitaxy.pm6y6.dos.urbach_energy('both', urbach_energy)
	Oghma.Epitaxy.update()

	Oghma.add_job(experiment_name)
	Oghma.run_jobs()
\end{minted}
\caption{Using PyOghma to run a simulation.} 
\label{python-example3}
\end{listing}

In this example PyOghma is imported as po, and a source OghmaNano json file is manipulated by changing the values of mobility, trap density, trapping rate and Urbach Energy. The original file is cloned with the line:

\begin{listing}[H]
\begin{minted}[frame=single,framesep=3mm,linenos=false,xleftmargin=21pt,tabsize=4]{python}
Oghma.clone('NewExperiment0')
\end{minted}
\end{listing}

Then at the end of the code the following two lines. The first of which adds the job to the job list in PyOghma. And the second line tells PyOghma to execute all the jobs. If there were more than one job PyOghma would execute multiple jobs across all CPUs, until they were all finished. If for example one wanted to run simulations with different values of mobility, one would add each simulation to the jobs list, then call $run\_jobs$ once to run the jobs across all cores in an efficient way.

\begin{listing}[H]
\begin{minted}[frame=single,framesep=3mm,linenos=false,xleftmargin=21pt,tabsize=4]{bash}
Oghma.add_job(experiment_name)
Oghma.run_jobs()
\end{minted}
\end{listing}



More information about PyOghma is available on the GitHub page.
