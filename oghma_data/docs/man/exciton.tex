\chapter{Modelling excitons/geminate recombination - organics only}
\section{Why you should not model excitons}
\label{sec:dont_do_excitions}
There are a number of models to calculate the number of geminate pairs which get converted to free charge carriers the Onsager-Braun model for example will give you the exciton dissociation efficiency.  There are other models which will enable you to calculate the distribution of excitons in a device as a function of position.  However, these models will generally require a number of parameters which are often not reliably known for a material system. Such parameters include exciton life-time, diffusion length and dissociation rate. So although it's possible (and interesting) to write a model to simulate geminate recombination, one is usually better off simply introducing a \emph{photon efficiency factor} $\eta_{photon}$. This number ranges between 0.0 and 1.0 and is multiplied by the number of photons absorbed at any point in the device to account for geminate recombination losses.

\begin{equation}
\label{eq:contn}
G=G_{abs}\cdot \eta_{photon}
\end{equation}

where $G$ is the charge carrier generation rate in $m^{-3}s^{-1}$ in equations \ref{eq:contn} and  \ref{eq:contp}.
 
This factor can be obtained to a reasonable degree by comparing the difference between the simulated and experimental $J_{sc}$.  This parameter can set in the configuration section of the optical simulation window. So therefore my advice is that in most cases you should not be modelling excitons explicitly but rather using the 'photon efficiency factor'. If you really want to model excitons read on..

\section{Modelling excitons}
\label{sec:excitions}
So if you have read section \ref{sec:dont_do_excitions} and still think you want to model excitons this section will explain how to do it.  Gvpdm includes an exciton solver. This sits between the optical model and electrical model.  If the exciton model is turned off then generation is simply the number of photons absorbed at any point in the device multiplied by the \emph{photon efficiency factor} see equation \ref{eq:contn}. If the exciton model is turned on then optical absorption will feed straight into the exciton diffusion equation.

\begin{equation}
\label{eq:exciton}
\frac{\partial X}{\partial t} = \nabla \cdot D \nabla X +G_{optical} -k_{dis} X -k_{FRET} X- k_{PL} X-\alpha X^2.
\end{equation}

where $X$ is the exciton density as a function of position, $D$ is the diffusion constant, $G_{optical}$ exciton generation rate. This value is taken straight from the optical model.  The constant $k_{dis}$ is exciton dissociation rate to free charge carriers.  When the exciton model is switched on $G$ in equations equals $k_{dis} X$. $k_{FRET}$ is the F\"{o}ster resonance energy transfer, $k_{PL} X$ is the radiative loss and $\alpha$ is an exciton-exciton annihilation rate constant.  The diffusion term is defined as 

\begin{equation}
\label{eq:exciton}
D=\frac{L^2}{\tau}
\end{equation}

Where $L$ is exciton diffusion length and $\tau$ is the exciton lifetime.

\section{Modeling excitions in a device}

\section{Modeling excitions in a unit cell}
