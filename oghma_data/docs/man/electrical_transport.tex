\section{Charge carrier transport}
To describe charge carrier transport, the bi-polar drift-diffusion equations are solved in position space
for electrons,
\begin{equation}
\label{eq:ndrive}
\boldsymbol{J_n} = q \mu_e n_{f}  {\nabla E_{c}}  + q D_n  {\nabla n_{f}},
\end{equation}
and holes,
\begin{equation}
\label{eq:pdrive}
\boldsymbol{J_p} = q \mu_h p_{f}  {\nabla E_{v}}  - q D_p {\nabla p_{f}}.
\end{equation}

Conservation of charge carriers is forced by solving the charge carrier continuity equations for both electrons,
\begin{equation}
\label{eq:contn}
\nabla \boldsymbol{J_n}  = q (R-G+\frac{\partial n}{\partial t}),
\end{equation}
and holes
\begin{equation}
\label{eq:contp}
\nabla \boldsymbol{J_p} = - q (R-G+\frac{\partial p}{\partial t}).
\end{equation}

where $R$ and $G$ are the net recombination and generation rates per unit volume respectively.
