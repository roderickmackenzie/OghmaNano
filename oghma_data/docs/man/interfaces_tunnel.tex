The equations below came from section 4.16.3.1 "Possible Conduction Mechanisms" of in the chapter "Electronic Properties of Alkanethiol Molecular Junctions: Conduction Mechanisms, Metal–Molecule Contacts, and Inelastic Transport" in the book, Comprehensive Nanoscience and Technology. They are referenced to Sze SM (1981) Physics of Semiconductor Devices.
%But I can not see their table in Sze

\subsection{Direct tunnelling}
\begin{equation}
\boldsymbol{J} = A(n-n^{eq}) Vexp  \left( -\frac{2d}{\hbar} \sqrt{2m q\phi}  \right)
\end{equation}
$A$ is a constant, $V$ is the applied bias, and $\phi$ is the barrier height calculated from the band structure, m is the mass of an electron, and d is the thickness of the barrier. In the model this is implemented as:
\begin{equation}
\boldsymbol{J} = A(n-n^{eq}) Vexp  \left( -B \sqrt{\phi}  \right)
\end{equation}

\subsection{Tunnelling organic-organic}
This is not classical tunnelling, but assumes the carriers can drift into trap states at the interface, it is really only applicable for organics.

Tunnelling of holes through hetrojunction interfaces are is give by
\begin{equation}
\boldsymbol{J_p} = q T_{h}  ((p_{1}-p_{1}^{eq})-(p_{0}-p_{0}^{eq})),
\end{equation}

and for electrons

\begin{equation}
\boldsymbol{J_n} = -q T_{e}  ((n_{1}-n_{1}^{eq})-(n_{0}-n_{0}^{eq})).
\end{equation}

Where $T_{h}$ and $T_{e}$ represent the rate constants of the tunnelling. This can be configured in the interfaces editor.

%%%%%%%%%%%%%%%%%
\subsection{Fowler–Nordheim tunnelling}
\begin{equation}
\boldsymbol{J} = A(n-n^{eq}) V^2 exp  \left( -\frac{q4d\sqrt{2m} \phi^{3/2}}{3q \hbar V}  \right)
\end{equation}
\emph{Not yet implemented but could be on request.} $A$ is a constant, $V$ is the applied bias, and $\phi$ is the barrier height calculated from the band structure, m is the mass of an electron, and d is the thickness of the barrier.  In the model this is implemented as:

\begin{equation}
\boldsymbol{J} = A(n-n^{eq}) V^2 exp  \left( -\frac{B \phi^{3/2}}{V}  \right)
\end{equation}

\subsection{Thermionic emission}
\begin{equation}
\boldsymbol{J} = A(n-n^{eq}) T^2 exp  \left( -\frac{q\phi -q\sqrt{qV/ 4 \pi \epsilon d}}{kT}  \right)
\end{equation}

\emph{Not yet implemented but could be on request.} $A$ is a constant, $V$ is the applied bias, and $\phi$ is the barrier height calculated from the band structure, m is the mass of an electron, and d is the thickness of the barrier.  In the model this is implemented as:

\begin{equation}
\boldsymbol{J} = A(n-n^{eq}) T^2 exp  \left( -\frac{q\phi -B\sqrt{V}}{kT}  \right)
\end{equation}

\subsection{Hopping conduction}
\emph{Not yet implemented but could be on request.}
\begin{equation}
\boldsymbol{J} = A(n-n^{eq}) V exp  \left( -\frac{q\phi}{kT}  \right)
\end{equation}
