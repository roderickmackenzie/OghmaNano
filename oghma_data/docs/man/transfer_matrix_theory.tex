\subsection{Theory of the transfer matrix method}  \label{ssec:transfer_matrix_theory}
On the left of the interface the electric field is given by

\begin{equation}
E_{1}=E^{+}_{1} e^{-j k_1 z}+E^{-}_{1} e^{j k_1 z}
\label{efield1}
\end{equation}
and on the right hand side of the interface the electric field is given by
\begin{equation}
E_{2}=E^{+}_{2} e^{-j k_2 z}+E^{-}_{2} e^{j k_2 z}
\label{efield2}
\end{equation}

Maxwel's equations give us the relationship between the electric and magnetic fields for a plane wave.

\begin{equation}
\nabla \times E=-j\omega \mu H 
\end{equation}
which simplifies to:
\begin{equation}
\frac{\partial E} {\partial z}=-j\omega \mu H 
\label{maxwel}
\end{equation}

Applying equation \ref{maxwel} to equations \ref{efield1}-\ref{efield2}, we can get the magnetic field on the left of the interface
\begin{equation}
-j \mu \omega H^{y}_{1}=-j k_1 E^{+}_{1} e^{-j k_1 z}+j k_1 E^{-}_{1} e^{j k_1 z}
\end{equation}
and on the right of the interface
\begin{equation}
-j \mu \omega H^{y}_{2}=-j k_2 E^{+}_{2} e^{-j k_2 z}+j k_2 E^{-}_{2} e^{j k_2 z}.
\end{equation}

Tidying up gives,
\begin{equation}
H^{y}_{1}=\frac{k}{\omega \mu}E^{+}_{1} e^{-j k_1 z}-\frac{k}{\omega \mu} E^{-}_{1} e^{j k_1 z}
\end{equation}

\begin{equation}
H^{y}_{2}=\frac{k}{\omega \mu}E^{+}_{2} e^{-j k_2 z}-\frac{k}{\omega \mu} E^{-}_{2} e^{j k_2 z}
\end{equation}

%%%%%%%%%%%
\subsubsection{Boundary conditions}
We now apply the electric and magnetic boundary conditions\cite{0953-8984-25-21-215301}
\begin{equation}
\mathbf{n} \times (\mathbf{E_2}-\mathbf{E_1})=0
\end{equation}

\begin{equation}
\mathbf{n} \times (\mathbf{H_2}-\mathbf{H_1})=0
\end{equation}

We let the interface be at z=0, which gives,
\begin{equation}
(E_{2}^{+}+E_{2}^{-})-(E_{1}^{+}+E_{1}^{-})=0
\label{electric_boundary}
\end{equation}
and
\begin{equation}
\frac{k_1}{\omega \mu}(E_{2}^{+}-E_{2}^{-})-(E_{1}^{+}-E_{1}^{-})\frac{k_2}{\omega \mu}=0
\end{equation}
.
The wavevector is given by
\begin{equation}
k=\frac{2 \omega }{\lambda}=\frac{\omega n}{c}
\end{equation}
.
We can therefore write the magnetic boundary condition as
\begin{equation}
n_2 (E_{2}^{+}-E_{2}^{-}) - n_1 (E_{1}^{+}-E_{1}^{-})=0
\label{mag_boundary}
\end{equation}

\subsubsection{Forward propagating wave}
Rearrange equation, \ref{mag_boundary} to give,

\begin{equation}
E_{1}^{-} = E_{1}^{+}-\frac{n_2}{n_1}(E_{2}^{+}-E_{2}^{-})
\end{equation}
Inserting in equation \ref{electric_boundary}, gives 
\begin{equation}
E_{2}^{+}+E_{2}^{-}=E_{1}^{+}+E_{1}^{+}-\frac{n_2}{n_1}(E_{2}^{+}-E_{2}^{-})
\end{equation}

\begin{equation}
2E_{1}^{+}=E_{2}^{+}+E_{2}^{-}+\frac{n_2}{n_1}(E_{2}^{+}-E_{2}^{-})
\end{equation}

\begin{equation}
2E_{1}^{+}\frac{n_1}{n_1+n_2}=E_{2}^{+}+E_{2}^{-}\frac{n_1-n_2}{n_1+n_2}
\end{equation}

\subsubsection{Backwards propagating wave}
Rearrange equation, \ref{mag_boundary} to give,

\begin{equation}
E_{1}^{+}=E_{1}^{-} +\frac{n_2}{n_1}(E_{2}^{+}-E_{2}^{-})
\end{equation}

Inserting in equation \ref{electric_boundary}, gives 
\begin{equation}
E_{2}^{+}+E_{2}^{-}=E_{1}^{-} +\frac{n_2}{n_1}(E_{2}^{+}-E_{2}^{-})+E_{1}^{-}
\end{equation}

\begin{equation}
2E_{1}^{-}=E_{2}^{+}+E_{2}^{-}- \frac{n_2}{n_1}(E_{2}^{+}-E_{2}^{-})
\end{equation}

\begin{equation}
2E_{1}^{-}\frac{n_1}{n_1+n_2}=E_{2}^{+}\frac{n_1-n_2}{n_1+n_2}+E_{2}^{-}
\end{equation}
Which is the same result as obtained in \cite{10.1063/1.1534621}.

These equations become:

\begin{equation}
E_{1}^{-}t_{12}=E_{2}^{+}r_{12}+E_{2}^{-}
\end{equation}

and
\begin{equation}
E_{1}^{+}t_{12}=E_{2}^{+}+E_{2}^{-}r_{12}
\end{equation}

Accounting for propagation we can write.  Note the change in sign between \cite{10.1063/1.1534621} and this work, this is because of how I have defined my wave equation. 
\begin{equation}
E_{1}^{+}t_{12}=E_{2}^{+}e^{\zeta_2 d_1}+E_{2}^{-}r_{12}e^{-\zeta_2 d_1}
\end{equation}
and

\begin{equation}
E_{1}^{-}t_{12}=E_{2}^{+}r_{12}e^{\zeta_2 d_1}+E_{2}^{-}e^{-\zeta_2 d_1}
\end{equation}

where
\begin{equation}
\zeta=\frac{2\pi}{\lambda} \bar{n}
\end{equation}

\input{optical_solver}

\subsection{Refractive index and absorption}
\begin{equation}
E(z,t)=Re(E_0 e^{j(-kz+\omega t)})= Re(E_0 e^{j(\frac{-2 \pi (n+j\kappa)}{\lambda}z + \omega t)})=e^{\frac{2\pi\kappa z}{\lambda}}Re(E_0 e^{\frac{j(-2 \pi (n+j\kappa)}{\lambda}z +\omega t})
\end{equation}
And because the intensity is proportional to the square of the electric field the absorption coefficient becomes

\begin{equation}
e^{-\alpha x}=e^{\frac{2\pi\kappa z}{\lambda}}
\end{equation}

\begin{equation}
\alpha=-\frac{4\pi\kappa}{\lambda_0}
\end{equation}


\newpage
\vfill
