
\chapter{Text past this point needs to be rewritten}


\subsubsection{Should I trust the results of gpvdm?}
Yes!  The model it's self has been verified against experiment [there are over 20 publications doing this, in steady state, time domain (us-fs time scales), and fx-domain]. The basic drift-diffusion solver was cross checked and compared against other drift diffusion models, and the accuracy compared down to 6-9 dp.  While the optical model has been compared to analytical solutions of Maxwell's equations.  The SRH model has also been compared against analytical models.  If the answers you are getting out of gpvdm are odd, then I would suggest to take a look at the input parameters.  If your efficienceis are high, try increasing the number of trap states, the recombination cross sections or reducing the e/h mobilites.  Finally, I would also recommend always running the latest version, and keeping an eye on the twitter stream for bug announcements.



\subsection{Can I use the model to simulate my exotic* material system/contacts?}
The short answer is yes.  The model is an effective medium model, meaning that it does not simulate the details of the medium, rather it approximates the medium with a set of electrical parameters.  For example, when simulating an organic solar cell, it does not simulate every detail of the BHJ, rather it just assumes an effective mobility, density of states, recombination cross sections, trapping cross sections and so on...  So if you can find electrical parameters to aproximate your material system (or guess them), there is nothing stopping you using gpvdm to simulate any exotic device/material.  The same goes for the contacts, the model simulates the contacts simply as a charge density. So if you have fancy graphene contacts which inject lots of charge, use a high majority carrier density on the contacts.  Where as if you have some dirty old ITO contacts may be drop the majority carrier density a bit.

