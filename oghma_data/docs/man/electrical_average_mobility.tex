\subsection{Average free carrier mobility}
In this model there are two types of electrons (holes), free electrons (holes) and trapped electrons (holes).  Free electrons (holes) have a finite mobility of $\mu_e^0$ ($\mu_h^0$) and trapped electrons (holes) can not move at all and have a mobility of zero.  To calculate the average mobility we take the ratio of free to trapped carriers and multiply it by the free carrier mobility.:

\begin{equation}
\mu_e(n)=\frac{\mu_e^0 n_{free}}{n_{free}+n_{trap}}
\end{equation}

Thus if all carriers were free, the average mobility would be $\mu_e^0$ and if all carriers were trapped the average mobility would be 0.  It should be noted that only $\mu_e^0$ ($\mu_h^0$) are used in the model for computation and $\mu_e(n)$ is an output parameter.

The value of $\mu_e^0$ ($\mu_h^0$) is an input parameter to the model.  This can be edited in the electrical parameter editor.  The value of $\mu_e(n)$, and $\mu_h(p)$ are output parameters from the model.  The value of $\mu_e(n)$, and $\mu_h(p)$ change as a function of position, within the device, as the number of both free and trapped charge carriers change as a function of position.  The values of  $\mu_e(x)$, and $\mu_h(x)$ can be found in $mu\_n\_ft.dat$ and $mu\_p\_ft.dat$ within the $snapshots$ directory.  The spatially averaged value of mobility, as a function of time or voltage can be found in the files $dynamic\_mue.dat$ or $dynamic\_muh.dat$ within the dynamic directory.

Were one to try to measure mobility using a technique such as CELIV or ToF, one would expect to get a value closer to $\mu_e(n)$ or $\mu_h(p)$ rather than closer to $\mu_e^0$ or $\mu_h^0$.  It should be noted however, that measuring mobility in disordered materials is a difficult thing to do, and one will get a different experimental value of mobility depending upon which experimental measurement method one uses, furthermore, mobility will change depending upon the charge density profile within the device, and thus upon the applied voltage and light intensity.  To better understand this, try for example doing a CELIV simulation, and plotting $\mu_e(n)$ as a function of time (Voltage).  You will see that mobility reduces as the negative voltage ramp is applied, this is because carriers are being sucked out of the device.  Then try extracting the mobility from the transient using the CELIV equation for extracting mobility.  Firstly, the CELIV equation will give you one value of mobility, which is a simplification of reality as the value really changes during the application of the voltage ramp.  Secondly, the value you get from the equation will almost certainly not match either $\mu_e^0$ or any value of $\mu_e(n)$.  This simply highlights, the difficult of measuring $a$ value of mobility for a disordered semiconductor and that really when we quote a value of mobility for a disordered material, it really only makes sense to quote a value measured under the conditions a material will be used.  For example, for a solar cell, values of $\mu_e(n)$ and $\mu_h(n)$, would be most useful to know under 1 Sun at the $P_{max}$ point on a JV curve.

